\documentclass[oneside]{book}
\usepackage{xcolor}


\newcommand{\exercisename}{Latihan}
\newcommand{\solutionname}{Solusi}

\definecolor{main}{RGB}{0,120,2}

%% Exercise with counter
\newcounter{exer}[chapter]
\setcounter{exer}{0}
\renewcommand{\theexer}{\thechapter.\arabic{exer}}
\newenvironment{exercise}[1][]{
  \refstepcounter{exer}
  \par\noindent\textbf{\color{main}{\exercisename} \theexer #1 }\rmfamily}{
  \par\ignorespacesafterend}

\newenvironment{solution}{\par\noindent\textbf{\color{main}\solutionname} \em}{\par}

\begin{document}


\chapter{Introduction}

% Pritami Sergio 1-4
\begin{exercise}
    Contoh soal 1
\end{exercise}

\begin{solution}
    Contoh solusi
\end{solution}

\vspace{12pt}

\begin{exercise}
  Contoh soal
\end{exercise}

\begin{solution}
  Contoh solusi
\end{solution}

\vspace{12pt}

\begin{exercise}
  Contoh soal
\end{exercise}

\begin{solution}
  Contoh solusi
\end{solution}

\vspace{12pt}

\begin{exercise}
  Contoh soal
\end{exercise}

\begin{solution}
  Contoh solusi
\end{solution}

\vspace{12pt}

% Muhammad Adi Syahputra 5-8
\begin{exercise}
  Contoh soal 5
\end{exercise}

\begin{solution}
  Contoh solusi
\end{solution}

\vspace{12pt}

\begin{exercise}
  Contoh soal
\end{exercise}

\begin{solution}
  Contoh solusi
\end{solution}

\vspace{12pt}

\begin{exercise}
  Contoh soal
\end{exercise}

\begin{solution}
  Contoh solusi
\end{solution}

\vspace{12pt}

\begin{exercise}
  Contoh soal
\end{exercise}

\begin{solution}
  Contoh solusi
\end{solution}

\vspace{12pt}

% Amelia Marta Dilova 9-12
\begin{exercise}
  Contoh soal 9
\end{exercise}

\begin{solution}
  Contoh solusi
\end{solution}

\vspace{12pt}

\begin{exercise}
  Contoh soal
\end{exercise}

\begin{solution}
  Contoh solusi
\end{solution}

\vspace{12pt}

\begin{exercise}
  Contoh soal
\end{exercise}

\begin{solution}
  Contoh solusi
\end{solution}

\vspace{12pt}

\begin{exercise}
  Contoh soal 12
\end{exercise}

\begin{solution}
  Contoh solusi
\end{solution}

\chapter{Network Model}

% Muhammad Riyadhtul Akbar 1-3
\begin{exercise}
  Contoh soal 1
\end{exercise}

\begin{solution}
  Contoh solusi
\end{solution}

\vspace{12pt}

\begin{exercise}
  Contoh soal
\end{exercise}

\begin{solution}
  Contoh solusi
\end{solution}

\vspace{12pt}

\begin{exercise}
  Contoh soal
\end{exercise}

\begin{solution}
  Contoh solusi
\end{solution}

\vspace{12pt}

% Muhammad Rizky Fadillah 4-7
\begin{exercise}
  Contoh soal 4
\end{exercise}

\begin{solution}
  Contoh solusi
\end{solution}

\vspace{12pt}

\begin{exercise}
  Contoh soal
\end{exercise}

\begin{solution}
  Contoh solusi
\end{solution}

\vspace{12pt}

\begin{exercise}
  Contoh soal
\end{exercise}

\begin{solution}
  Contoh solusi
\end{solution}

\vspace{12pt}

\begin{exercise}
  Contoh soal
\end{exercise}

\begin{solution}
  Contoh solusi
\end{solution}

\vspace{12pt}

% Rizky Sandiary 8-11
\begin{exercise}
  Contoh soal 8
\end{exercise}

\begin{solution}
  Contoh solusi
\end{solution}

\vspace{12pt}

\begin{exercise}
  Contoh soal
\end{exercise}

\begin{solution}
  Contoh solusi
\end{solution}

\vspace{12pt}

\begin{exercise}
  Contoh soal
\end{exercise}

\begin{solution}
  Contoh solusi
\end{solution}

\vspace{12pt}

\begin{exercise}
  Contoh soal
\end{exercise}

\begin{solution}
  Contoh solusi
\end{solution}

\chapter{Data and Signals}

% Dimas Yediberto Luciano Dien 1-3
\begin{exercise}
  Contoh soal 1
\end{exercise}

\begin{solution}
  Contoh solusi
\end{solution}

\vspace{12pt}

\begin{exercise}
  Contoh soal
\end{exercise}

\begin{solution}
  Contoh solusi
\end{solution}

\vspace{12pt}

\begin{exercise}
  Contoh soal
\end{exercise}

\begin{solution}
  Contoh solusi
\end{solution}

\vspace{12pt}

% Karel Chavez H 4-6
\begin{exercise}
Berapakah lebar bandwidth suatu sinyal yang dapat diuraikan menjadi lima gelombang sinus dengan frekuensi 0, 20, 50, 100, dan 200 Hz? Semua amplitudo puncak adalah sama. Gambarkan bandwidthnya.
\end{exercise}

\begin{solution}
220 ns = 220 x 10 -9 s = apakah bandwidth suatu sinyal dapat diuraikan menjadi lima gelombang sinus dengan frekuensi 0, 20, 50, 100, dan 200 Hz? Semua amplitudo puncak adalah sama.
\end{solution}

\vspace{12pt}

\begin{exercise}
Sinyal composite signal dengan bandwidth 2000 Hz terdiri dari dua gelombang sinus. Yang pertama memiliki frekuensi 100 Hz dengan amplitudo maksimum 20 V; yang kedua memiliki amplitudo maksimum 5 V. Gambarkan bandwidthnya.
\end{exercise}

\begin{solution}
bandwidth = Fh - Fl, bandwith = 2000 , Terendah = 100 , Tertinggi = 2100 Fh - Fl = 2100 - 100 ,bandwith = 2000
\end{solution}

\vspace{12pt}

\begin{exercise}
Sinyal manakah yang memiliki bandwidth lebih lebar, gelombang sinus dengan frekuensi 100 Hz atau gelombang sinus dengan frekuensi 200 Hz?
\end{exercise}

\begin{solution}
setiap sinyal adalah sinyal sederhana dalam hal ini. Bandwidth dari sinyal sederhana adalah 0.jadi, bandwidth dari kedua sinyal r sama.
\end{solution}

\vspace{12pt}

% Muhammad Arie Setya Putra Pala 7-9
\begin{exercise}
\\
Apa hubungan teorema Nyquist dengan komunikasi?
\end{exercise}

\begin{solution}
\\
Teorema Nyquist-Shannon juga dikenal sebagai teorema pengambilan sampel adalah ketentuan fisik mendasar untuk komunikasi di mana sinyal kontinu dalam waktu terkait dengan sinyal diskrit dalam waktu. Ini pada dasarnya menetapkan jumlah pengambilan sampel minimum yang memungkinkan urutan diskrit untuk menangkap semua sinyal kontinu.
\end{solution}

\vspace{12pt}

\begin{exercise}
\\
Apa hubungan kapasitas Shannon dengan komunikasi?
\end{exercise}

\begin{solution}
\\
Batas Shannon atau kapasitas Shannon dari saluran komunikasi mengacu pada tingkat maksimum data bebas kesalahan yang secara teoritis dapat ditransfer melalui saluran jika tautan mengalami kesalahan transmisi data acak, untuk tingkat kebisingan tertentu.
\end{solution}

\vspace{12pt}

\begin{exercise}
\\
Mengapa sinyal optik yang digunakan pada kabel serat optik memiliki panjang gelombang yang sangat pendek?
\end{exercise}

\begin{solution}
Optical signals have very high frequencies. A high frequency means a short wave length because the wave length is inversely proportional to the frequency.
\end{solution}

\vspace{12pt}

% Ricky 10-12
\begin{exercise}
\\
Bisakah kita mengatakan jika suatu sinyal periodik atau nonperiodik hanya dengan melihat frekuensinya petak domain ? bagaimana ?
\end{exercise}

\begin{solution}
\\
  bisa, karena sinyal periodik dapat dilihat dari frekuensinya yang memiliki periode waktu dasar berulang pada interval waktu yang teratur sedangkan sinyal non-periodik itu acak dan tidak dapat di definisi seperti pada gelombang sinus atau gelombang kosinus.
\end{solution}

\vspace{12pt}

\begin{exercise}
\\
Apakah plot domain frekuensi dari sinyal suara itu diskrit atau kontinu?
\end{exercise}

\begin{solution}
\\
Domain frekuensi sinyal suara biasanya kontinu karena suara adalah sinyal nonperiodik.
\end{solution}

\vspace{12pt}

\begin{exercise}
\\
Apakah plot domain frekuensi dari sistem alarm itu diskrit atau kontinu?
\end{exercise}

\begin{solution}
\\
Sistem alarm biasanya periodik. Oleh karena itu, plot domain frekuensinya adalah diskrit.
\end{solution}

\vspace{12pt}

% Fajar Bimantara 13-16
\begin{exercise}
  Contoh soal 13
\end{exercise}

\begin{solution}
  Contoh solusi
\end{solution}

\vspace{12pt}

\begin{exercise}
  Contoh soal
\end{exercise}

\begin{solution}
  Contoh solusi
\end{solution}

\vspace{12pt}

\begin{exercise}
  Contoh soal
\end{exercise}

\begin{solution}
  Contoh solusi
\end{solution}

\vspace{12pt}

\begin{exercise}
  Contoh soal
\end{exercise}

\begin{solution}
  Contoh solusi
\end{solution}

\vspace{12pt}

% Julicko Pratama Putra 17-19
\begin{exercise}
  Contoh soal 17
\end{exercise}

\begin{solution}
  Contoh solusi
\end{solution}

\vspace{12pt}

\begin{exercise}
  Contoh soal
\end{exercise}

\begin{solution}
  Contoh solusi
\end{solution}

\vspace{12pt}

\begin{exercise}
  Contoh soal
\end{exercise}

\begin{solution}
  Contoh solusi
\end{solution}

\vspace{12pt}

% Nadjamudin Beda 20-23
\begin{exercise}
  Contoh soal 20
\end{exercise}

\begin{solution}
  Contoh solusi
\end{solution}

\vspace{12pt}

\begin{exercise}
  Contoh soal
\end{exercise}

\begin{solution}
  Contoh solusi
\end{solution}

\vspace{12pt}

\begin{exercise}
  Contoh soal
\end{exercise}

\begin{solution}
  Contoh solusi
\end{solution}

\vspace{12pt}

\begin{exercise}
  Contoh soal
\end{exercise}

\begin{solution}
  Contoh solusi
\end{solution}

\chapter{Digital Transmission}

% Pritami Sergio 1-4
\begin{exercise}
  Contoh soal 1
\end{exercise}

\begin{solution}
  Contoh solusi
\end{solution}

\vspace{12pt}

\begin{exercise}
  Contoh soal
\end{exercise}

\begin{solution}
  Contoh solusi
\end{solution}

\vspace{12pt}

\begin{exercise}
  Contoh soal
\end{exercise}

\begin{solution}
  Contoh solusi
\end{solution}

\vspace{12pt}


\begin{exercise}
  Contoh soal
\end{exercise}

\begin{solution}
  Contoh solusi
\end{solution}

\vspace{12pt}

% Muhammad Adi Syahputra 5-8
\begin{exercise}
  Contoh soal 5
\end{exercise}

\begin{solution}
  Contoh solusi
\end{solution}

\vspace{12pt}

\begin{exercise}
  Contoh soal
\end{exercise}

\begin{solution}
  Contoh solusi
\end{solution}

\vspace{12pt}

\begin{exercise}
  Contoh soal
\end{exercise}

\begin{solution}
  Contoh solusi
\end{solution}

\vspace{12pt}

\begin{exercise}
  Contoh soal
\end{exercise}

\begin{solution}
  Contoh solusi
\end{solution}

\vspace{12pt}

% Amelia Marta Dilova 9-12
\begin{exercise}
  Contoh soal 9
\end{exercise}

\begin{solution}
  Contoh solusi
\end{solution}

\vspace{12pt}

\begin{exercise}
  Contoh soal
\end{exercise}

\begin{solution}
  Contoh solusi
\end{solution}

\vspace{12pt}

\begin{exercise}
  Contoh soal
\end{exercise}

\begin{solution}
  Contoh solusi
\end{solution}

\vspace{12pt}

\begin{exercise}
  Contoh soal
\end{exercise}

\begin{solution}
  Contoh solusi
\end{solution}

\vspace{12pt}

% Muhammad Riyadhtul Akbar 13-16
\begin{exercise}
  Contoh soal 13
\end{exercise}

\begin{solution}
  Contoh solusi
\end{solution}

\vspace{12pt}

\begin{exercise}
  Contoh soal
\end{exercise}

\begin{solution}
  Contoh solusi
\end{solution}

\vspace{12pt}

\begin{exercise}
  Contoh soal
\end{exercise}

\begin{solution}
  Contoh solusi
\end{solution}

\vspace{12pt}

\begin{exercise}
  Contoh soal
\end{exercise}

\begin{solution}
  Contoh solusi
\end{solution}

\vspace{12pt}

% Muhammad Rizky Fadillah 17-20
\begin{exercise}
  Contoh soal 17
\end{exercise}

\begin{solution}
  Contoh solusi
\end{solution}

\vspace{12pt}

\begin{exercise}
  Contoh soal
\end{exercise}

\begin{solution}
  Contoh solusi
\end{solution}

\vspace{12pt}

\begin{exercise}
  Contoh soal
\end{exercise}

\begin{solution}
  Contoh solusi
\end{solution}

\vspace{12pt}

\begin{exercise}
  Contoh soal
\end{exercise}

\begin{solution}
  Contoh solusi
\end{solution}

\chapter{Analog Transmission}

% Rizky Sandiary 1-4
\begin{exercise}
  Calculate the baud rate for the given bit rate and type of modulation.
  \begin{itemize}
    \item[a.] 2000 bps, FSK
    \item[b.] 4000 bps, ASK
  \end{itemize}
\end{exercise}

\begin{solution}
  We use the formula $S = (1/r) \times N$, but first we need to calculate the value of r for each case.
  \begin{itemize}
    \item[a.] $r = log_22 = 1 \quad \rightarrow \quad S = (1/1) \times (2000 \textnormal{ bps}) = 2000 \textnormal{ baud}$
    \item[b.] 
  \end{itemize}
\end{solution}

\vspace{12pt}

\begin{exercise}
  Contoh soal
\end{exercise}

\begin{solution}
  Contoh solusi
\end{solution}

\vspace{12pt}

\begin{exercise}
  Contoh soal
\end{exercise}

\begin{solution}
  Contoh solusi
\end{solution}

\vspace{12pt}

\begin{exercise}
  Contoh soal
\end{exercise}

\begin{solution}
  Contoh solusi
\end{solution}

\vspace{12pt}

% Dimas Yediberto Luciano Dien 5-8
\begin{exercise}
  Contoh soal 5
\end{exercise}

\begin{solution}
  Contoh solusi
\end{solution}

\vspace{12pt}

\begin{exercise}
  Contoh soal
\end{exercise}

\begin{solution}
  Contoh solusi
\end{solution}

\vspace{12pt}

\begin{exercise}
  Contoh soal
\end{exercise}

\begin{solution}
  Contoh solusi
\end{solution}

\vspace{12pt}

\begin{exercise}
  Contoh soal
\end{exercise}

\begin{solution}
  Contoh solusi
\end{solution}

\vspace{12pt}

% Karel Chavez H 9-12
\begin{exercise}
Sebuah perusahaan memiliki media dengan bandwidth 1-MHz (lowpass). Korporasi perlu membuat 10 saluran independen terpisah yang masing-masing mampu mengirim setidaknya 10 Mbps. Perusahaan telah memutuskan untuk menggunakan teknologi QAM. Berapa jumlah bit minimum per baud untuk setiap saluran? Berapa jumlah titik dalam diagram konstelasi untuk setiap saluran? Misalkan d = O.
\end{exercise}

\begin{solution}
Pertama, kami menghitung bandwidth untuk setiap saluran = (1 MHz) / 10 = 100 KHz. Kami kemudian menemukan nilai r untuk setiap saluran: B = (1 + d) × (1/r) × (N) →r = N / B →r = (1 Mbps/100 KHz) = 10 Kemudian kita dapat menghitung jumlah level: L = 2r = 210 = 1024. Ini berarti bahwa kita memerlukan teknik 1024-QAM untuk mencapai kecepatan data ini.
\end{solution}

\vspace{12pt}

\begin{exercise}
Manakah dari empat teknik konversi digital-ke-analog (ASK, FSK, PSK atau QAM) yang paling rentan terhadap noise? Pertahankan jawaban Anda.
\end{exercise}

\begin{solution}
Menurut saya, ASK adalah teknik yang paling rentan di antara empat teknik konversi digital ke analog. Karena amplitudo lebih dipengaruhi oleh noise daripada fasa atau frekuensi.
\end{solution}

\vspace{12pt}

\begin{exercise}
Definisikan konversi analog-ke-analog?
\end{exercise}

\begin{solution}
Proses mengubah salah satu karakteristik sinyal analog untuk mewakili amplitudo sesaat dari sinyal baseband disebut konversi analog-ke-analog. Ini juga disebut modulasi sinyal analog sinyal analog broadband dasar memodulasi pembawa untuk membuat sinyal analog broadband.
\end{solution}

\vspace{12pt}

\begin{exercise}
Manakah dari tiga teknik konversi analog-ke-analog (AM, FM, atau PM) yang paling rentan terhadap noise? Pertahankan jawaban Anda.
\end{exercise}

\begin{solution}
Menurut saya, AM, FM, PM, di antara ketiga teknik konversi analog ke analog ini, teknik yang paling rentan adalah AM karena amplitudo lebih dipengaruhi oleh noise daripada fase atau frekuensi.
\end{solution}

\chapter{Bandwidth Utilization: Multiplexing and Spreading}

% Muhammad Arie Setya Putra Pala 1-4
\begin{exercise}
\\
Jelaskan tujuan dari multiplexing
\end{exercise}

\begin{solution}
\\
Tujuan multiplexing adalah untuk memungkinkan sinyal ditransmisikan lebih efisien melalui saluran komunikasi tertentu, sehingga mengurangi biaya transmisi.
\end{solution}

\vspace{12pt}

\begin{exercise}
\\
Sebutkan tiga teknik multiplexing utama yang disebutkan dalam bab ini.
\end{exercise}

\begin{solution}
\\
frequency-division multiplexing (FDM), wave-division multiplexing (WDM), and time-division multiplexing (TDM).
\end{solution}

\vspace{12pt}

\begin{exercise}
\\
Bedakan antara tautan dan saluran dalam multiplexing.
\end{exercise}

\begin{solution}
\\
Dalam multiplexing, kata link mengacu pada jalur fisik. Kata saluran mengacu pada bagian dari tautan yang membawa transmisi antara sepasang garis tertentu. Satu tautan dapat memiliki banyak (n) saluran.
\end{solution}

\vspace{12pt}

\begin{exercise}
\\
Manakah dari tiga teknik multiplexing yang digunakan untuk menggabungkan sinyal analog?
Manakah dari tiga teknik multiplexing yang digunakan untuk menggabungkan sinyal digital?
\end{exercise}

\begin{solution}
\\
FDM dan WDM digunakan untuk menggabungkan sinyal analog; bandwidth dibagi. TDM digunakan untuk menggabungkan sinyal digital; waktunya dibagi.
\end{solution}

\vspace{12pt}

% Ricky 5-7
\begin{exercise}
\\
Tentukan hierarki analog yang digunakan oleh perusahaan telepon dan buat daftar level hierarki yang berbeda.
\end{exercise}

\begin{solution}
\\
Hirarki analog menggunakan saluran suara (4 KHz), grup (48 KHz), grup super (240 KHz), grup master (2,4 MHz), dan grup jumbo (15,12 MHz). \\
Struktur analog tertentu menggunakan saluran distribusi kata. (kelas, kelompok, kelas reli, jumbogroup).
\end{solution}

\vspace{12pt}

\begin{exercise}
\\
Tentukan hierarki analog yang digunakan oleh perusahaan telepon dan buat daftar level hierarki yang berbeda.
\end{exercise}

\begin{solution}
\\
Hirarki analog menggunakan saluran suara (4 KHz), grup (48 KHz), grup super (240 KHz), grup master (2,4 MHz), dan grup jumbo (15,12 MHz). \\
Struktur analog tertentu menggunakan saluran distribusi kata. (kelas, kelompok, kelas reli, jumbogroup).
\end{solution}

\vspace{12pt}

\begin{exercise}
\\
Manakah dari tiga teknik multiplexing yang umum untuk link serat optik? Jelaskan alasannya.
\end{exercise}

\begin{solution}
\\
WDM umum untuk multiplexing sinyal optik karena memungkinkan multiplexing sinyal dengan frekuensi yang sangat tinggi.
\end{solution}

\vspace{12pt}

% Fajar Bimantara 8-11
\begin{exercise}
  Contoh soal 8
\end{exercise}

\begin{solution}
  Contoh solusi
\end{solution}

\vspace{12pt}

\begin{exercise}
  Contoh soal 9
\end{exercise}

\begin{solution}
  Contoh solusi
\end{solution}

\vspace{12pt}

\begin{exercise}
  Contoh soal
\end{exercise}

\begin{solution}
  Contoh solusi
\end{solution}

\vspace{12pt}

\begin{exercise}
  Contoh soal
\end{exercise}

\begin{solution}
  Contoh solusi
\end{solution}

\vspace{12pt}

% Julicko Pratama Putra 12-14
\begin{exercise}
  Contoh soal 12
\end{exercise}

\begin{solution}
  Contoh solusi
\end{solution}

\vspace{12pt}

\begin{exercise}
  Contoh soal
\end{exercise}

\begin{solution}
  Contoh solusi
\end{solution}

\vspace{12pt}

\begin{exercise}
  Contoh soal
\end{exercise}

\begin{solution}
  Contoh solusi
\end{solution}

\vspace{12pt}

% Nadjamudin Beda 15-18
\begin{exercise}
  Contoh soal 15
\end{exercise}

\begin{solution}
  Contoh solusi
\end{solution}

\vspace{12pt}

\begin{exercise}
  Contoh soal
\end{exercise}

\begin{solution}
  Contoh solusi
\end{solution}

\vspace{12pt}

\begin{exercise}
  Contoh soal
\end{exercise}

\begin{solution}
  Contoh solusi
\end{solution}

\vspace{12pt}

\begin{exercise}
  Contoh soal
\end{exercise}

\begin{solution}
  Contoh solusi
\end{solution}

\end{document}

